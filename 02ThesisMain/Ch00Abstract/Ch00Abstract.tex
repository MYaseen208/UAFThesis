
\chapter*{Abstract}
\noindent
Genotypes of different genetic structure behave differently in different environmental conditions. 
Genotype-by-environment interaction (GEI) is referred as differential responses of different genotypes across different environments; GEI has great importance because of higher performance of genotypes to be assessed by GEI. But presence of GEI makes analysis more complicated. To up-root these assessment complications several methods have been proposed such as Principal Component Analysis (PCA), Cluster Analysis, Additive Main effects and Multiplicative Interaction(AMMI) models  and Genotype plus Genotype by Environment interaction (GGE). These methods neither overcome the problem of over parameterization nor use the prior information. The aim of this study is to use such technique which can address these problems. For this purpose a wheat crop data comprised of 30 genotypes test across 13 different locations of punjab, Pakistan for two consecutive years was used. The layout of the experiment was Randomized complete Block Design(RCBD). In this study a comparison was made between Classical methods AMMI, GGE biplot and Bayesian approach using Von-Mises Fisher distribution as prior. classical methods showed that genotype V-11098 was the most desirable genotype based on stability and high yield performance. Bayesian approach was used for GEI because it makes statistical interpretation rather easy by relaxing some constraints and it uses the prior information, also provides solution for these by using MCMC algorithm. Bayesian strategy for analysis of GEI was used to assess the general, specific performance of genotypes and risk related to genotype. Analysis revealed that bilinear terms $u_{25,1}$ for genotype NS-10 genotype and $v_{13,1}$ for environment S13 (Piplan-14) were found significant indicated that these have effect on interaction. It was observed that Bayesian approach can nicely explore GE interaction.
