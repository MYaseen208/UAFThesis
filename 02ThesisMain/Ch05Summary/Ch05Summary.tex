\chapter{Summary}\label{summary_chapter}
The objective of the study were (a): To examine the Genotype environment interaction using Bayesian approach; (b): Graphical represent of bi-plots using prior information; (c): To elaborate that Bayesian models can be easily adapted to GEI. For this purpose an experiment was conducted, which consist of 30 genotypes of wheat in thirteen different location of Punjab for two consecutive years. The experiment was arranged in RCBD layout with three replication.

Firstly combined analysis of variance was and it was observed that genotype and environment  main effect as well as interaction was found significant. Then AMMI ANOVA suggested by \cite{GOLLOB1968} was performed, analysis revealed that PC1 explain 23.4 and Pc2 explain 14.7 variation for interaction sum of square.

Bayesian approach using von-Mises Fisher distribution was applied to elicit prior for first year data. Credible Regions was obtained and bilinear terms which do not contain null points (0, 0) were refereed significant. Bilpots using first and second bilinear terms were drawn and observed that bilinear terms  $u_{25,1}$ for genotype NS-10 genotype and $v_{13,1}$ for environment S13 (Piplan-14) have significant effect on interaction. From result it is concluded that Bayesian bilinear model can be applied to yield trial data because of following advantages.
 \begin{itemize}
 	
 	\item  Confidence regions for genotype and environments for parameters and for their interaction can be obtain naturally.
 	\item  This approach provide facilitation to identify genotypes and environments that have impact on significant interaction; furthermore, genotypes and environments having similar responses can also be recognized. 
 	\item It can deal with unbalanced data (most of the time present in yield trials)  naturally.
 	\item  Prior information related to genotypes, environments and for interaction can be incorporated by using Bayesian approach for means as well as variances.
 	\item This approach  can be adopted for cells having unequal size.
 
 \end{itemize}
 
 Although there is plenty of literature available related to Bayesian frame-work but a little work has been done on Bayesian inference for yield trails. It is suggested this methodology can be carried out other bilinear model by minimizing some restrictions and putting some parameters equal to zero. This approach will prove a good alternative of Classical method for Analyzing Genotype by environment interaction, which is often a major area of concern for yield experimenters.